\documentclass{article}
\usepackage{amsmath}
\newcommand{\mean}[1]{\ensuremath{\langle #1 \rangle}}
\title{Notes on the Stretched Exponential Distribution}
\author{Ben Gamari}
\begin{document}
\section{Notes on the Stretched Exponential Distribution}

The power exponential distribution is given by,

\[ p(\tau \mid \lambda, \beta) d\tau
   = \beta \tau^{\beta-1} \lambda^{\beta} e^{-(\tau \lambda)^\beta} d\tau\]

Note that with $\beta = 1$ we recover the exponential distribution. as
$\beta \rightarrow 2$ this approaches the normal distribution with
$\mean{\tau} = 0$, $\mean{\tau^2} = ?$. The case where $\beta < 1$ is
known as a stretched exponential.

The first and second moments of this distribution are given by,

\begin{align*}
  \mean{\tau} & = \lambda^{-1} \beta^{-1} \Gamma(\beta^{-1}) \\
  \mean{\tau^2} & = 2 \lambda^{-2} \beta^{-1} \Gamma(2\beta^{-1})  \\
\end{align*}

The maximum likelihood estimator for the $\lambda$ parameter given $\beta$ is,
\[ \hat \lambda^\beta = \frac{1}{N} \sum_{i=1}^N (\tau_i^\beta - \tau_N^\beta) \]
Where $\tau_1 > \tau_2 > ... > \tau_N$ are the $N$ largest observed samples.

The maximum likelihood estimator for $\beta$ is,
\[
  \hat\beta^{-1} = \frac{\sum_{i=1}^N (\tau_i^\beta \ln\tau_i -  \tau_N^\beta \ln\tau_N)}
                       {\sum_{i=1}^N (\tau_i^\beta - \tau_N^\beta)}
                  - \frac{1}{N} \sum_{i=1}^N \ln\tau_i
\]
Being an implicit function, one must generally solve for $\beta$ by numerical means.

\subsection{References}
\begin{itemize}
\item D. Sornette. \it{Critical Phenomena in Natural Sciences: Chaos, Fractals, Selforganization and Disorder} 
\end{itemize}

\end{document}
